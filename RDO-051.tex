\documentclass[OPS,toc]{lsstdoc}
% lsstdoc documentation: https://lsst-texmf.lsst.io/lsstdoc.html

% Generated by Makefile
\input{meta}

% Package imports go here.

% Local commands go here.

% If you want glossaries, uncomment:
% \input{aglossary.tex}
% \makeglossaries

\title{Users Committee Charge}
% \setDocSubtitle{Optional subtitle}

\author{%
Michael Strauss and the Rubin Science Advisory Council
}

\setDocRef{RDO-051}
\setDocUpstreamLocation{\url{https://github.com/lsst/RDO-051}}
\date{\vcsDate}
% \setDocCurator{The Curator of this Document}

\setDocAbstract{%
This document contains the charge for the Rubin Observatory Users Committee.
}

% Revision history.
% Order: oldest first.
% Fields: VERSION, DATE, DESCRIPTION, OWNER NAME.
% See LPM-51 for version number policy.
\setDocChangeRecord{%
  \addtohist{0}{2021-08-06}{Draft.}{Science Advisory Committee}
  \addtohist{1}{2021-09-10}{Version 1}{Science Advisory Committee}
  \addtohist{1}{2021-12-29}{Version 1.1}{Melissa Graham added time committment details.}
}

\begin{document}

\maketitle

During Rubin Observatory operations, the Legacy Survey of Space and Time (LSST) will produce a range of data products, made available to the Rubin data rights community through the Rubin Science Platform (RSP).
The Rubin Community Engagement Team (CET) will be seeking regular feedback from the science user community on the quality of the data products and the efficacy of the RSP throughout the lifetime of the LSST.
The Rubin Users Committee is charged both with soliciting this feedback from the community, and actively using the LSST data and the RSP.
Users Committee members may also participate in User Acceptance Testing for the RSP under the guidance of the CET.
The Users Committee will regularly report to the Rubin Lead Community Scientist and to the Rubin Observatory Director with recommendations for science-driven improvements to the LSST data products and the RSP tools and services.
Their goal is to maximize the scientific productivity of the user community, while prioritizing equitable access and inclusive practices throughout the diverse user community. 

The Users Committee will be a standing committee in place by fall 2021, and will continue through the life of Rubin Observatory operations.
Before the start of the 10-year survey, the focus of the committee will be on the Data Previews (which will include simulation data, precursor survey data, and commissioning data), and community involvement thereof.
Once the 10-year LSST is underway, the focus of the committee will shift to use of the RSP, the regular data releases, the alert stream, and the prompt data products.  

\section{Committee Membership}

The Users Committee shall\footnote{In this document, “shall” or “will” refers to a requirement and “should” refers to a strong recommendation.} consist of twelve members, all of whom must have full LSST data rights.
The membership should include at least two scientists from Chile and six scientists from the US.
It should also include at least one person the United Kingdom, one from France CC-IN2P3, and one from a smaller international member country.
Of the six members from the US, one should be from a small US institution, and one should be (or become) a member of the NOIRLab CSDC Users Committee.
Individuals who are NOIRLab, SLAC, or Rubin Observatory employees are not eligible for membership. 

The expertise of User Committee members shall cover the breadth of the LSST core science themes: Dark Matter and Dark Energy, the Transient Universe, the Structure of the Milky Way, and the Inventory of the Solar System.
If possible, it should also cover the full richness of other science areas that Rubin will impact, for example galaxy evolution and AGN.
However, User Committee members will not serve either as direct representatives of specific Science Collaborations or of science communities (e.g., LSST-UK).
In fact, there is no requirement that User Committee members be members of LSST Science Collaborations.
The members should be representative of the diverse membership of the Rubin Observatory data rights community, including race and ethnicity, gender, career stage, and type and location of institution of employment (the latter as described in the previous paragraph).

As the Users Committee will convey detailed feedback on the LSST data products and the RSP from the science community, members should include active and experienced users of Rubin Observatory data products or relevant precursors from other surveys or Rubin-oriented simulations, and/or the Rubin science pipelines.
They are expected to maintain their expertise with the Rubin Observatory data products and the RSP.
Users Committee membership appointments shall be for two years. 
Members may be asked to serve another year, for up to four years total, when needed (e.g., to fulfill the membership requirements of the Users Committee).

In 2021, the Rubin Science Advisory Committee (SAC) will suggest a slate of candidates to serve as initial members on the Users Committee; candidates will be approved by the Operations Director.
The SAC will identify one of those candidates to serve as the initial Chair of the Users Committee.
The Chair will serve for two years; Users Committee members themselves will be responsible for selecting one of their own to serve as Chair after the initial Chair’s term ends.
A given individual may serve two terms (i.e., four years) as Chair.
The Chair is the principal point of contact with the CET.
The CET and Users Committee Chair together will arrange meetings of the committee.
The Chair will ensure that the minutes of the meetings are prepared and posted publicly. 

Individuals may be nominated for membership on the Users Committee by filling out a form (see Appendix \ref{sec:apA}) and e-mailing it to the Chair of the Rubin Observatory SAC.
Self-nominations are  encouraged.
Once a year the CET will provide a list of which Users Committee members are rolling off the committee and must be replaced and which are available to serve another year.
The SAC will review the list of nominations and put forward a slate to the Operations Director, who will then approve the list.

Users Committee members shall identify themselves as such when appropriate, e.g., in public forums or when discussing aspects of the Rubin Observatory with colleagues, so that their role is clear, and they are visible and accessible to the broader community.
They shall agree to have their contact information posted and to receive correspondence from Rubin users.

\section{Committee Activities}

The Users Committee shall hold at least two meetings yearly.
These can be held virtually or face-to-face (with virtual conferencing enabled for those unable to travel).
A natural venue for a face-to-face meeting would be in coordination with the annual Rubin Observatory Project and Community Workshop (PCW).
Meetings shall be open to all members of the LSST user community, but the members of the Users Committee can call for closed executive sessions as needed.  

The Lead Community Scientist or their designee, and a representative of the Operations leadership team or their designee, shall attend each of these meetings in an \textit{ex officio} capacity.
Additional members of the CET or the Rubin Project team shall be invited to attend, as suggested by the Users Committee Chair.
At these meetings, representatives of the CET will present updates or demonstrations of new features or aspects of the data products and/or the RSP.
The Committee shall prepare reports with recommendations for changes, upgrades, or new developments to the LSST data products and/or the RSP, send them to the CET and the Operations Director, and post them on a public website (e.g., \url{http://community.lsst.org}).
The CET will respond to these recommendations (e.g., indicating which recommendations the Observatory will be able to incorporate, and on what timescale) at the subsequent meeting of the User’s Committee.  

The CET, in coordination with the Users Committee, will actively solicit feedback on LSST data products and the RSP from the community, via an on-line portal and project-wide communication tools (e.g., \url{http://community.lsst.org}), and by being available to talk directly with people (e.g., at workshops).
With the assistance of the CET, they may conduct polls of the LSST user community regarding scientific aspects of the Rubin Observatory data products and services.

Users Committee members are expected to ingest and relay feedback from the user community no matter the mode in which it was given (informal discussion, formal correspondence).
This should be done in a way to maintain confidentiality of the user unless that confidentiality is explicitly waived. 

In between their meetings, the User Committee members will be invited to participate in User Acceptance Testing activities for the RSP under the guidance of the CET.  

The Chair of the Users Committee shall also serve as an ex officio member of the Users Committee of the National Optical-Infrared Astronomical Research Laboratory (NOIRLab). 

\subsection{Time Commitment}

The expected time commitment for UC members is no more than 20 working hours per semester (5 days per year).

Every 6 months, all UC members would spend up to 4 hours in the UC meetings and another $\sim$4 hours preparing the UC report to Rubin.
The other 12 "UC service hours" per semester would be spent, for example: preparing and analyzing surveys of the user base; corresponding with users who contact them (e.g., replying to unsolicited feedback); relaying emergent issues to the CET for follow-up; and participating in User Acceptance Testing for the RSP. 
It is likely that UC members would spend different fractions of their time on the above activities, as suits their experience and expertise.

UC members are encouraged to track their hours spent and to not exceed 20 hours per semester, and to let the CET know if that becomes insufficient. 

Activities that inform a UC member's participation in the UC, but would be considered a part of their regular scientific endeavors and not counted as "UC service hours", might include, for example: using the RSP and the LSST data products; engaging in Q\&A and discussions on the Community Forum; attending the annual week-long Rubin Project and Community Workshop; and interacting with other LSST users at scientific conferences.

\section{Rubin Support for the Users Committee}

The activities of the Users Committee will be coordinated and facilitated by the CET.

This includes:
\begin{itemize}
\item enabling communications (Slack channels, mailing lists),
\item arranging and hosting meetings,
\item seeking Rubin expertise on technical issues that may arise,
\item providing tutorials for the Committee regarding new features,
\item developing an on-line portal for comments from the community,
\item organizing surveys or polls for community feedback at the request of the Committee, 
\item and suggesting timely topics for the Committee’s consideration.
\end{itemize}

The CET will also guide Users Committee members to participate in User Acceptance Testing activities.
Travel funding for 12 people in support of at least one of the two yearly meetings being face-to-face are included in the non-labor costs of the Community Engagement Team.
Virtual attendance will be an option for all meetings.
While service on the Users Committee is not currently a paid position, the work done on the Committee is a vital and significant contribution to the operations of Rubin Observatory, and will be recognized as such by the Rubin leadership. 

\clearpage

\appendix

\section{Nomination Form}\label{sec:apA}

Please create the nomination form as a document (preferably a PDF) with the following components and email it to Michael Strauss\footnote{strauss at astro dot princeton dot edu}, chair of the Rubin Observatory Science Advisory Council (SAC).
The contents of these nomination forms will be seen only by the Rubin SAC members and the Rubin Observatory operations director.
Self-nominations are encouraged!

\begin{enumerate}
\item Nominee name.
\item Nominee email.
\item Nominee institutional affiliation.
\item Nominated by (if not a self-nomination).
\item Career stage that best describes the nominee for the next two years (e.g., graduate student; postdoctoral researcher; research staff scientists / engineer; junior / senior / emeritus professor).
\item Science Collaborations (if any) the nominee is a member of.
\item Nominee’s Rubin-related science interests (up to one paragraph).
\item Nominee’s experience with the Rubin Science Platform or LSST Science Pipelines; their relevant experience with the Rubin Observatory data products and services such as Data Preview 0 or Stack Club; their unique perspective (e.g., junior researcher, historically underrepresented in astronomy); and/or their technical experience with relevant precursor surveys such as, e.g., HSC, KIDS, or DES (up to one page).
\end{enumerate}

\clearpage

% Include all the relevant bib files.
% https://lsst-texmf.lsst.io/lsstdoc.html#bibliographies
\section{References} \label{sec:bib}
\renewcommand{\refname}{} % Suppress default Bibliography section
\bibliography{local,lsst,lsst-dm,refs_ads,refs,books}

% Make sure lsst-texmf/bin/generateAcronyms.py is in your path
\section{Acronyms} \label{sec:acronyms}
\input{acronyms.tex}
% If you want glossary uncomment below and comment out the two lines above.
% \printglossaries

\end{document}
